\documentclass[final,9pt,fleqn]{beamer}\usepackage[]{graphicx}\usepackage[]{color}
%% maxwidth is the original width if it is less than linewidth
%% otherwise use linewidth (to make sure the graphics do not exceed the margin)
\makeatletter
\def\maxwidth{ %
  \ifdim\Gin@nat@width>\linewidth
    \linewidth
  \else
    \Gin@nat@width
  \fi
}
\makeatother

\definecolor{fgcolor}{rgb}{0.345, 0.345, 0.345}
\newcommand{\hlnum}[1]{\textcolor[rgb]{0.686,0.059,0.569}{#1}}%
\newcommand{\hlstr}[1]{\textcolor[rgb]{0.192,0.494,0.8}{#1}}%
\newcommand{\hlcom}[1]{\textcolor[rgb]{0.678,0.584,0.686}{\textit{#1}}}%
\newcommand{\hlopt}[1]{\textcolor[rgb]{0,0,0}{#1}}%
\newcommand{\hlstd}[1]{\textcolor[rgb]{0.345,0.345,0.345}{#1}}%
\newcommand{\hlkwa}[1]{\textcolor[rgb]{0.161,0.373,0.58}{\textbf{#1}}}%
\newcommand{\hlkwb}[1]{\textcolor[rgb]{0.69,0.353,0.396}{#1}}%
\newcommand{\hlkwc}[1]{\textcolor[rgb]{0.333,0.667,0.333}{#1}}%
\newcommand{\hlkwd}[1]{\textcolor[rgb]{0.737,0.353,0.396}{\textbf{#1}}}%
\let\hlipl\hlkwb

\usepackage{framed}
\makeatletter
\newenvironment{kframe}{%
 \def\at@end@of@kframe{}%
 \ifinner\ifhmode%
  \def\at@end@of@kframe{\end{minipage}}%
  \begin{minipage}{\columnwidth}%
 \fi\fi%
 \def\FrameCommand##1{\hskip\@totalleftmargin \hskip-\fboxsep
 \colorbox{shadecolor}{##1}\hskip-\fboxsep
     % There is no \\@totalrightmargin, so:
     \hskip-\linewidth \hskip-\@totalleftmargin \hskip\columnwidth}%
 \MakeFramed {\advance\hsize-\width
   \@totalleftmargin\z@ \linewidth\hsize
   \@setminipage}}%
 {\par\unskip\endMakeFramed%
 \at@end@of@kframe}
\makeatother

\definecolor{shadecolor}{rgb}{.97, .97, .97}
\definecolor{messagecolor}{rgb}{0, 0, 0}
\definecolor{warningcolor}{rgb}{1, 0, 1}
\definecolor{errorcolor}{rgb}{1, 0, 0}
\newenvironment{knitrout}{}{} % an empty environment to be redefined in TeX

\usepackage{alltt}
\setbeamertemplate{bibliography item}[text]
\setbeamertemplate{caption}[numbered]
\usepackage{url,lmodern}
\usepackage{amsmath,amsfonts,amssymb,graphicx,
  amsthm,subfigure,multirow,hyperref,textcomp,natbib}
\usepackage[iso,american]{isodate}
\usepackage{listings}
\hypersetup{colorlinks=true,urlcolor=cyan,linkcolor=cyan}
\usepackage{tikz}
\usetikzlibrary{arrows,automata,shapes,chains}

\defbeamertemplate*{footline}{infolines theme} { \leavevmode%
  \hbox{%
    \begin{beamercolorbox}[wd=.333333\paperwidth,ht=2.25ex,dp=1ex,center]{author
        in head/foot}%
      \usebeamerfont{author in
        head/foot}\insertshortauthor%~~(\insertshortinstitute)
    \end{beamercolorbox}%
    \begin{beamercolorbox}[wd=.333333\paperwidth,ht=2.25ex,dp=1ex,center]{title
        in head/foot}%
      \usebeamerfont{title in head/foot}\insertshorttitle
    \end{beamercolorbox}%
    \begin{beamercolorbox}[wd=.333333\paperwidth,ht=2.25ex,dp=1ex,right]{date
        in head/foot}%
      \usebeamerfont{date in head/foot}\insertshortdate{}\hspace*{2em}
      \insertframenumber{} / \inserttotalframenumber\hspace*{2ex}
    \end{beamercolorbox}}%
  \vskip0pt%
}

\usepackage{eso-pic}
\setlength{\paperwidth}{11in}
\setlength{\paperheight}{7in}
\setlength{\mathindent}{5pt}
\setbeamerfont{frametitle}{size=\Huge}
\setbeamertemplate{navigation symbols}{}
\setbeamertemplate{footline}{\hfill {\footnotesize \href{https://creativecommons.org/licenses/by-sa/4.0/}{CC BY-SA 4.0} $\circ$ Edward A. Roualdes $\circ$ \href{https://roualdes.us}{roualdes.us} $\circ$ version $0.0.1$ $\circ$ updated: \today} \hspace {0.1in} \vspace{0.1in}}
\IfFileExists{upquote.sty}{\usepackage{upquote}}{}
\begin{document}



\newcommand\AtPagemyUpperLeft[1]{\AtPageLowerLeft{%
    \put(\LenToUnit{0.81\paperwidth},\LenToUnit{0.925\paperheight}){#1}}}
\AddToShipoutPictureFG{
  \AtPagemyUpperLeft{{
      \href{http://dplyr.tidyverse.org}{\includegraphics[scale=0.2,keepaspectratio]{dplyr.png}}
      \hspace{0.025in}
      \href{http://ggplot2.tidyverse.org}{\includegraphics[scale=0.2,keepaspectratio]{ggplot.png}}
      \hspace{0.025in}
      \href{https://www.r-project.org/}{\includegraphics[scale=0.05,keepaspectratio]{Rlogo.png}}
      \hspace{0.025in}
      \href{https://www.csuchico.edu}{\raisebox{-0.1\height}{\includegraphics[scale=0.13]{chico_seal.eps}}}
}}
}%



\begin{frame}[fragile]
  \frametitle{ \underline{Analysis of Variance} :: {\Large \textbf{CHEAT SHEET}} }
  \hspace{0.1in}
  \begin{columns}

    \begin{column}{0.03\paperwidth}
    \end{column}

    \begin{column}{0.3\paperwidth}
      \begin{block}{{\huge Basics}}

        Are the means amongst the $k$ levels equal? \\
        \vspace{-0.1in}
        \begin{align*}
          H_0: \quad  & \mu_1 = \mu_2 = \mu_3 = \ldots = \mu_K \\
          H_1: \quad  & \text{at least one mean is different} \\
          \alpha = 0.01 &  \\
        \end{align*}
      \end{block}
      \vspace{-0.25in}
      Step $1$, plot the data!
      \vspace{-0.1in}


\begin{knitrout}
\definecolor{shadecolor}{rgb}{0.969, 0.969, 0.969}\color{fgcolor}\begin{kframe}
\begin{alltt}
\hlkwd{ggplot}\hlstd{(df,} \hlkwd{aes}\hlstd{(Factor, Numerical))} \hlopt{+}
    \hlkwd{geom_jitter}\hlstd{(}\hlkwc{width}\hlstd{=}\hlnum{0.1}\hlstd{,} \hlkwc{colour}\hlstd{=}\hlstr{"grey75"}\hlstd{)} \hlopt{+}
    \hlkwd{stat_summary}\hlstd{(}\hlkwc{fun.data}\hlstd{=}\hlstr{"mean_cl_boot"}\hlstd{,}
                 \hlkwc{colour}\hlstd{=}\hlstr{"cornflowerblue"}\hlstd{)}
\end{alltt}
\end{kframe}
\includegraphics[width=\maxwidth]{figures/main-unnamed-chunk-1-1} 

\end{knitrout}

\vspace{-0.1in}
\begin{knitrout}\large
\definecolor{shadecolor}{rgb}{0.969, 0.969, 0.969}\color{fgcolor}\begin{kframe}
\begin{alltt}
\hlstd{fit} \hlkwb{<-} \hlkwd{lm}\hlstd{(Numerical} \hlopt{~} \hlstd{Factor,} \hlkwc{data}\hlstd{=df)}
\hlkwd{print}\hlstd{(}\hlkwd{anova}\hlstd{(fit),} \hlkwc{signif}\hlstd{=}\hlnum{FALSE}\hlstd{)}
\end{alltt}
\begin{verbatim}
## Analysis of Variance Table
## 
## Response: Numerical
##            Df Sum Sq Mean Sq F value Pr(>F)
## Factor      3     14    4.67    2.71  0.048
## Residuals 109    187    1.72
\end{verbatim}
\end{kframe}
\end{knitrout}
\vspace{-0.1in}

  \begin{block}{{\huge Assumptions}}

  \begin{enumerate}
  \item independent observations $y_{nk}$ for $n=1, \ldots, N$,
  \item within each group $k=1, \ldots, K$, data are normal, and
  \item the variability of each group is equal.
  \end{enumerate}

  \end{block}
\end{column}

\begin{column}{.03\paperwidth}
\end{column}

\begin{column}{0.3\paperwidth}

  \begin{block}{{\huge Intuition}}
    Generally, means will be different when the average square error amongst the groups is significantly larger than average square error within the groups.
  \end{block}


\begin{knitrout}
\definecolor{shadecolor}{rgb}{0.969, 0.969, 0.969}\color{fgcolor}
\includegraphics[width=\maxwidth]{figures/main-unnamed-chunk-3-1} 
\includegraphics[width=\maxwidth]{figures/main-unnamed-chunk-3-2} 

\end{knitrout}
\vspace{-0.2in}

\begin{centering}
\[ F_{K-1, N-K} = \frac{\text{average square error amongst the groups}}{\text{average square error within the group}} \]
\end{centering}



\begin{block}{{\huge F distribution}}

  The $F$-distribution is a probability density function over non-negative numbers.  P-values are strictly calculated from the right tail.  Large $F$ statistics indicate evidence against the null hypothesis.

\begin{knitrout}
\definecolor{shadecolor}{rgb}{0.969, 0.969, 0.969}\color{fgcolor}

{\centering \includegraphics[width=\maxwidth]{figures/main-unnamed-chunk-4-1} 

}



\end{knitrout}
\end{block}

\vspace{-0.2in}

\begin{block}{{\huge Details}}

  ANOVA decomposes observation $n$ into three pieces: the first group's mean $\alpha$, the offset $\beta_k$ for group $k$, and an error term $\epsilon_{nk}$.  Each $x_{nk}$ is a numerically encoded, binary variable that indicates when observation $n$ belongs to group $k$.
  \vspace{-0.15in}

  \[ y_{nk} = \alpha + \beta_1  x_{n1} + \ldots + \beta_{K-1} x_{n,K-1} + \epsilon_{nk}, \quad \epsilon_{nk} ~ \mathcal{N}(0, \sigma^2)\]
\end{block}

\end{column}

\begin{column}{0.03\paperwidth}
\end{column}

\begin{column}{0.25\paperwidth}

  \begin{block}{{\huge Coefficients}}
    The coefficients $\beta_k$ are group offsets relative to the first group: group $k$'s mean is equal to $\hat{\alpha} + \hat{\beta_k}$.
\begin{knitrout}
\definecolor{shadecolor}{rgb}{0.969, 0.969, 0.969}\color{fgcolor}\begin{kframe}
\begin{verbatim}
##           [,1]
## alpha   4.8607
## beta_1 -0.4670
## beta_2 -0.9337
## beta_3 -0.4795
\end{verbatim}
\end{kframe}
\end{knitrout}
  \end{block}

  \begin{block}{{\huge Group Means}}
    The expected value of $y_{nk}$, $E(y_{nk}|x_{nk})$, is found by averaging over the observations within group $k$.  If observation $n=5$ belongs to group $k=3$ then

    \begin{align*}
      \hat{y}_{5,3} = E(y_{5,3}|x_{5,3} == 1) & = \hat{\alpha} + \hat{\beta_3} \\
                              & = 4.8607 + -0.4795 \\
                              & = 3.927 \\
    \end{align*}

    Notice $\hat{y}_{5,3}$ is exactly equal to group $3$'s mean.

\begin{knitrout}
\definecolor{shadecolor}{rgb}{0.969, 0.969, 0.969}\color{fgcolor}\begin{kframe}
\begin{alltt}
\hlstd{df} \hlopt
    \hlkwd{group_by}\hlstd{(Factor)} \hlopt
    \hlkwd{summarise}\hlstd{(}\hlkwc{group_mean}\hlstd{=}\hlkwd{mean}\hlstd{(Numerical))}
\end{alltt}
\begin{verbatim}
## # A tibble: 4 x 2
##   Factor group_mean
##   <fctr>      <dbl>
## 1      1      4.861
## 2      2      4.394
## 3      3      3.927
## 4      4      4.381
\end{verbatim}
\end{kframe}
\end{knitrout}

  \end{block}
\end{column}


\end{columns}
\end{frame}
\end{document}
